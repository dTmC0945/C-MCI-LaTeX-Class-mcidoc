%

\chapter{Literature Review} %------------------------------------------------------------------
 

The term "high-speed" generally is used to define a motor that goes faster than 150 m/s
[2]. The mentioned speed can be achieved with a simple 2p = 2, 50 Hz machine with
around 1m diameter. Since the European grid is fed with 50 Hz AC it can achieve
only 3000 min-1. American grid can achieve 3600 min-1 since it uses 60 Hz AC. The
term has not have a solid cut-off point since some manufacturers put the limit on
3600 min-\footnote[2]{tt}
.       
1
The research in the field of high-speed machines has been mostly active in
Finland.[3] worked on ferromagnetic core matericals in smooth solid-rotors.[4]'s
topic was rotor designs and voltage sources suitable for high-speed machines. His
main research focus was the design of squirrel cage and coated solid rotors. [5]
studied high-speed induction machines thermal analysis and [6] was focused on air–
gap friction in high-speed machines.[7] and [8] worked on active magnetic bearings
used in high-speed induction machines. It must me noted that the research mentioned
above used machines that run on 100 Hz to 300 Hz.
Other dissertations have also been done on the field of solid rotor 
technology.[9] has studied on experimentally slitted solid rotors with 19 kW, 50 Hz,
2p = 4 induction motor. [10 11 12] has investigated the effects of axial slits, end
rings and cage windings in a solid ferromagnetic rotor with the values of 3 Hp
(around 2.23kW), 50 Hz, 2p = 6. Balarama Murt [13] has also made investigations
on axial slits on solid steel rotors. Woolley [14] studied new designs of unlaminated
rotors. Zaim [15] studied rotor concepts for induction motors.

In recent years the laboratory of electrical engineering at Lappeenranta University of
Technology (LUT) has done research on improving the efficiency of high-speed
solid-rotor construction. From this research it was found that when a solid-rotor is
used, the flux density distribution of the rotor surface plays a significant role. To
have the least amount of lossess, the rotor must have a perfectly sinusoidal rotor
surface. This also applies to time dependent and spatial harmonics. Research has
given promising results and the efficiency of these types of machines have increased
to the level comperable to a regular 3000 induction motor of same power
output.



  

The research at LUT got off with the study of a 12 kW, 400 Hz induction
motor [15]. Continuing from the research different types of rotor with different rotor
coatings and end rings with new stator design were used to improve the efficiency
[16]. From the successful results, 16 kW, 225 Hz induction motor with a smooth, a
slitted and a squirrel-cage solid rotors were experimented [17].
After these successful tests the focus was shifted to bigger machines. A 200
kW,140 Hz slitted Solid-Rotor induction machine and a 250 kW, 140 Hz slitted solid
rotor with copper end rings were analyzed [18].
It is known that by axially slitting the motor, its electromagnetic properties
can be improved. This type of construction must be done in an optimal way. The
penetratino of the flux lines must be calculated and this is directly connected to the
slit width and depth. According to [19]'s reseach, the optimal values for the rotor is
half of the radius for the depth and 1,5 mm width.
The output torque of an electric machine is related and proportional to the
product of the Amperé-turns and the magnetic flux per pole. Because for a given
motor size the Amperé-turns and the magnetic flux per pole are limited, the most
efficient way to increase the output is to increase the speed.
According to [20] there is a correlation between eddy current lossess and the
coating material for CCIM. This research show that by using a material the is at least
twice as conductive as the coated material the eddy lossess are dramatically reduced.
The thickness for the coating is also an improtant design factor. As it is used to
concentrate eddy currents to the surface a proximal value must be chosen.[21] shows
that by picking a thickness of 1,5 mm copper material an optimal performance can be
gained.


The most attractive advantage of high-speed range is the reduction or the
motor size. The volume per power ratio and the weight per power ratio are inversely
proportional to the rotating speed in high-speed.
Solid-rotors are also used for their mechanical properties. This type of rotor
are the strongest type one can construct. It can also be used in very high-speeds since
it can maintain its balance well. When a load is directly attached to the shaft, the
solid rotor can still show an impressive mechanical durability and can avoid
vibrations.High-speed solid rotor induction motors may be used in applications from
a few kilowatts up to tens of megawatts. Their main area are where the laminated
rotor construction are not durable enough. [2] has defined the speed limits for certain
rotor.

The mechanical limit of laminated rotor rotational speed varies from 10.000
min-1 50.000 min-1 (to reach these speeds special constructions may be
demanded)
 The mechanical limit of solid-rotor rotational speed varies from 20.000 min-1
to 100.000 min-1 (to reach these speeds special constructions may be
demanded)

\begin{minted}{cpp}   
  // C++ program to find all string
  // which are greater than given length k 
  
  #include <bits/stdc++.h> 
  using namespace std;
  
  // function find string greater than
  // length k
  void string_k(string s, int k) 
  {
    // create an empty string
    string w = "";
    // iterate the loop till every space
    for (int i = 0; i < s.size(); i++) {
      if (s[i] != ' ')
      
      // append this sub string in
      // string w
      w = w + s[i];
      else {
        
        // if length of current sub
        // string w is greater than
        // k then print
        if (w.size() > k)
        cout << w << " ";
        w = "";
      }
    }
  }
\end{minted}



%%% Local Variables:
%%% mode: LaTeX
%%% TeX-master: "../Vorlage_thesis"
%%% End:
